%%%%%%%%%%%%%%%%%%%%%%%%%%%%%%%%%%%%%%%%%%%%%%%%%%%%%%%%%%%%%%%%%%%
%% 
%% Yisong Yue's resume
%%   - based off work by Michael DeCorte 
%%
%%%%%%%%%%%%%%%%%%%%%%%%%%%%%%%%%%%%%%%%%%%%%%%%%%%%%%%%%%%%%%%%%%%



%%
%% The following code sets up the document formatting
%%

%this assumes that res_yy.sty is in some path
\documentstyle[hyperref, margin, line]{res_yy}

\hypersetup{backref,pdfpagemode=Full,colorlinks=true,backref}

\addtolength{\oddsidemargin}{-0.45in}
\addtolength{\voffset}{-0.30in}
\addtolength{\textwidth}{1.00in} 
%\addtolength{\textheight}{1.50in}

\renewcommand{\namefont}{\LARGE\emph}



%%
%% The following code defines some macros for terms which have raised font
%% (ie 4\fourth would result 4th with the 'th' raised (superscripted)
%%

\def\Cplusplus{{\rm C\raise.5ex\hbox{\small ++}}}
\def\CSharp{{\rm C\raise.5ex\hbox{\small \#}}}
% 'st' 'nd' 'rd' 'th' superscripts for numbers
\def\first{{\raise.5ex\hbox{\small st}}}
\def\second{{\raise.5ex\hbox{\small nd}}}
\def\third{{\raise.5ex\hbox{\small rd}}}
\def\fourth{{\raise.5ex\hbox{\small th}}}



%%
%% starting the actual document
%%

\begin{document}

%the name in big fonts at the top of resume
%this is left aligned
\name{Crist\'obal Mackenzie}

%this is right aligned
\address{
%website: www.yisongyue.com 
\ \ \ \ \ email: cmackenz@uc.cl  
}

\begin{resume}



%%
%% This section of code is inelegant, but I'm too lazy to fix it
%%

\section{\textsc{Research Interest}}
My work as a MSc student is currently focused on applying Machine Learning to astronomical survey datasets.

\section{\textsc{Education}}

\textbf{Pontificia Universidad Cat\'olica de Chile} \ \ \ \ \ \ \ \ \ \ \ \ \ \ \ \ \ \ \ \ \ \ \ \ \ \ \ \ \ \ \ \ \ \ \ \ \ \ \ \ \ \ \ \ \ \ \ \ \ 2013 - 2014 \\
M.Sc. in Computer Science

\textbf{Pontificia Universidad Cat\'olica de Chile} \ \ \ \ \ \ \ \ \ \ \ \ \ \ \ \ \ \ \ \ \ \ \ \ \ \ \ \ \ \ \ \ \ \ \ \ \ \ \ \ \ \ \ \ \ \ \ \ \ 2009 - 2014 \\ 
Industrial Engineering and Computer Science, GPA 5.66/7


%%
%% the meat of the resume starts now
%%

\begin{formatb}
  \employer{l}\title{r}\\
  \location{l}\dates{r}\\
  \body\\
\end{formatb}

\section{\textsc{Work Experience}}

\employer{\textbf{Google, Inc.}}
\title{Software Engineering Intern}
\location{Mountain View, CA}
\dates{January 2013 - March 2013}
\begin{position}
Developed a PubSub subscriber to help in the maintaining of indexes.
\end{position}

\employer{\textbf{Hadza}}
\title{Software Engineer}
\location{Santiago, Chile}
\dates{August 2011 - May 2012}
\begin{position}
Developed video recording and sharing app. Incubated in StartUp-Chile and AngelPad.
\end{position}

\employer{\textbf{Huntcha}}
\title{Software Engineer}
\location{Santiago, Chile}
\dates{June 2012 - August 2012}
\begin{position}
Developed Push notifications system.
\end{position}

\employer{\textbf{Freelance}}
\title{Software Engineer}
\location{Santiago, Chile}
\dates{August 2012 - November 2012}
\begin{position}
Developed time tracking app for Chilean SaaS company Lemontech.
\end{position}

% \employer{\textbf{Tap Pharmaceutical}}
% \title{Web Programmer}
% \location{Bannockburn, IL}
% \dates{July 10\fourth , 2001 - August 15\fourth, 2001}
% \begin{position}
% Wrote an online help reference for a web-based Documentum
% interface. Used Flash.
% \end{position}



%%
%% We use the same formatting for projects as for work experience
%% Shown below is the formatting used previously
%%
%%  \begin{formatb}
%%    \employer{l}\title{r}\\
%%    \location{l}\dates{r}\\
%%    \body\\
%%  \end{formatb}
%%
%% 
%%  Note that \location is now being used for non-location information
%%


%\begin{formatb}
%  \employer{l}\dates{r}\\
%  \body\\
%\end{formatb}

%\section{\textsc{Projects}}

%\employer{\textbf{An SVM Approach for Diversified Recommendations}}
%\dates{Fall 2007}
%\begin{position}
%Developed novel SVM approach to optimized a parameterized class of submodular functions for diversified retrieval.
%\end{position}

%\employer{\textbf{Algorithms on GPU}}
%\dates{Spring 2004}
%\begin{position}
%Surveyed existing GPGPU algorithms and implemented two dimensional FFT on image data using an NVIDIA
%Geforce FX 5600.
%\end{position}

%%
%% This section could also use more formatting, but looks ok, as is
%%

%\section{\textsc{Qualifications}}

%\emph{Programming Languages}: \Cplusplus, \CSharp, Cg, HLSL, ARB assembly, SML, OCaML, PHP, MySQL, Java, Python, Perl, MIPS assembly

%\emph{Libraries and Tools}: Vim, STL, DirectX, OpenGL, \LaTeX, GIMP, Adobe Suite, Macromedia Suite, MatLab, Mathematica, Microsoft Visual Studio, GCC, GDB


%%
%% Note that we're redefining the formatting
%% We only have one row of information now, instead of two
%%

\section{\textsc{Activities}}

\begin{formatb}
  \employer{l}\dates{r}\\
  \body\\
\end{formatb}

\employer{\textbf{Teaching Assistant}}
\dates{2010 - 2012}
\begin{position}
Teaching assistant on Introduction to Economics, Technology Entrepreneurship and Data Structures and Algorithms.
\end{position}

\employer{\textbf{Google Student Ambassador}}
\dates{March 2013 - Present}
\begin{position}
Promoting Google events, products and other opportunities like summer internships in campus.
\end{position}


%%
%% Nothing special here, just a normal table
%%

\section{\textsc{Course Work}}
  \begin{tabular}{lll}
  Advanced Programming   & \ \ & Software Engineering   \\ 
  Computer Architecture      & \ \ & Data Structures \& Algorithms   \\
  Operating Systems and Computer Networks   & \ \ & Software System Architecture   \\
  Discrete Mathematics    & \ \ & Calculus I, II \& III   \\
  Detailed Software Design     & \ \ & Software Development  \\
  Information Systems     & \ \ & Databases   \\
  Stochastic Models      & \ \ & Linear Algebra  \\
  Probabilistic Graphical Models      & \ \ &  \\
  \end{tabular}


\end{resume}
\end{document}
